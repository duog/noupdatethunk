\documentclass{beamer}


\newcommand{\hide}{\onslide+<+(1)->}

\usepackage[utf8]{inputenc}
\usepackage[TS1,T1]{fontenc}
\usepackage[english]{babel}
\usepackage{listings}
\usepackage{xcolor}
\usepackage{tikz}
\usetikzlibrary{backgrounds}
\usepackage{mathpartir}

\usetheme[titlepage0]{KIT}


\definecolor{light-gray}{gray}{0.95}
\lstdefinestyle{haskell}{
        ,columns=flexible
        ,basewidth={.365em}
        ,keepspaces=True
	,belowskip=0pt
	,backgroundcolor=\color{light-gray}
	,frame=single
	,xleftmargin=2em
	,xrightmargin=2em
        ,basicstyle=\small\sffamily
        ,stringstyle=\itshape
	,language=Haskell
        ,texcl=true
        ,showstringspaces=false
        ,keywords={module,where,import,data,let,in,case,of}
}
\lstnewenvironment{haskell}{\lstset{style=haskell}}{}


\title{dup -- Explicit un-sharing in Haskell}
\subtitle{Haskell Implementors Workshop 2012 -- Lightning Talk}
\KITtitleimage[page=3,viewport=190 110 550 400]{dup-hiw-2012}
\author{Joachim Breitner}
\date{2012-09-14}
\iflanguage{ngerman}{%
  \institute{LEHRSTUHL PROGRAMMIERPARADIGMEN}%
}{%
  \institute{PROGRAMMING PARADIGMS GROUP}%
}


\begin{document}

\maketitle

\begin{frame}
\frametitle{My motivation}
\begin{center}
\parbox{.5\linewidth}{
\centering
We need to provide our programmers with better tools to\\[0.5em]
\textbf{analyze}\\[0.5em]
and\\[0.5em]
\textbf{control}\\[0.5em]
the space behaviour of their Haskell programs.
}
\end{center}
\end{frame}


\begin{frame}[fragile,t]
\frametitle{Sharing can cause space leaks}

\begin{haskell}
let xs = [1..100000000]
in (last xs, length xs)
\end{haskell}
\vfill
\hide

\centering 
\begin{tikzpicture}
[level/.style={sibling distance=20mm/2^#1,level distance=6mm},
every node/.style={circle,inner sep=1.5pt},
eval/.style={circle,fill=black,color=black},
%every path/.style={->},
]
\path
  (0,0) node (comma) {(,)}
  (-1,-1) coordinate (ct1)
  (1,-1) coordinate (ct2)
  (0,-3) coordinate (clt);
\onslide<2>{\node at (ct1) (t1) {T};}
\onslide<3-5>{\node[eval] at (ct1) (t1) {:};}
\onslide<6->{\node at (ct1) (t1) {1e8};}

\onslide<2->{\node at (ct2) (t2) {T};}

\onslide<2-3>{\node at (clt) (lt) {T};}
\onslide<4->{\node[alias=cons0] at (clt) (lt) {:};}
\onslide<4>{
	\path (cons0) +(0,-1) node (n) {1};
	\draw (cons0) -- (n);
	\path (cons0) +(1,0) node (cons1) {T};
	\draw (cons0) -- (cons1);
}
\onslide<5->{
	\onslide<5-7>{
	\path (cons0) +(0,-1) node (n) {1};
	\draw (cons0) -- (n);
	\path (cons0) +(1,0) node (cons1) {:};
	\draw (cons0) -- (cons1);

	\path (cons1) +(0,-1) node (n) {2};
	\draw (cons1) -- (n);
	\path (cons1) +(1,0) node (cons2) {:};
	\draw (cons1) -- (cons2);

	\path (cons2) +(0,-1) node (n) {3};
	\draw (cons2) -- (n);
	\path (cons2) +(1,0) node (cons3) {:};
	\draw (cons2) -- (cons3);
	}

	\path (cons3) +(0,-1) node (n) {4};
	\draw (cons3) -- (n);
	\onslide<4>{\path (cons3) +(1,0) node (cons4) {T};}
	\onslide<5->{\path (cons3) +(1,0) node (cons4) {$\cdots$};}
	\draw (cons3) -- (cons4);
	}
\draw (comma) -- (t1);
\draw (comma) -- (t2);
\onslide<2-4>{\draw (t1) -- (lt);}
\onslide<5>{\draw (t1) -- (cons3);}
\draw (t2) -- (lt);
\end{tikzpicture}

\vfill
\onslide<7->{the programmer might want to avoid to have the list shared}

\end{frame}

\begin{frame}[fragile,t]
\frametitle{Source transformations may help}

\begin{haskell}
let xs () = [1..100000000]
in (last $ xs (), length $ xs ())
\end{haskell}
\vfill

\centering
\begin{tikzpicture}
[level/.style={sibling distance=20mm/2^#1,level distance=6mm},
every node/.style={circle,inner sep=1.5pt},
eval/.style={circle,fill=black,color=black},
%every path/.style={->},
]
\path
  (0,0) node (comma) {(,)}
 +(-1,-1) coordinate (ct1)
 +(1,-1) coordinate (ct2)
 +(0,-2) coordinate (cf);
\onslide<1>{\node at (ct1) (t1) {T};}
\onslide<2-5>{\node[eval] at (ct1) (t1) {:};}
\onslide<6->{\node at (ct1) (t1) {1e8};}

\onslide<1->{\node at (ct2) (t2) {T};}

\onslide<1->{\node at (cf) (f) {F};}
\path (t1) +(1,-2) coordinate (clt);
\onslide<3>{\path (clt) node (lt) {T};}
\onslide<4>{\node[alias=cons0] at (clt) (lt) {:};}
\onslide<5-6>{\node[alias=cons0,color=gray] at (clt) (lt) {:};}
\onslide<4>{
	\path (cons0) +(0,-1) node (n) {1};
	\draw (cons0) -- (n);
	\path (cons0) +(1,0) node (cons1) {T};
	\draw (cons0) -- (cons1);
}
\onslide<5-6>{
	\begin{scope}[color=gray]
	\path (cons0) +(0,-1) node (n) {1};
	\draw (cons0) -- (n);
	\path (cons0) +(1,0) node (cons1) {:};
	\draw (cons0) -- (cons1);

	\path (cons1) +(0,-1) node (n) {2};
	\draw (cons1) -- (n);
	\path (cons1) +(1,0) node (cons2) {:};
	\draw (cons1) -- (cons2);

	\path (cons2) +(0,-1) node (n) {3};
	\draw (cons2) -- (n);
	\end{scope}

	\onslide<5>{
	\path (cons2) +(1,0) node (cons3) {:};
	\draw[color=gray] (cons2) -- (cons3);
	\path (cons3) +(0,-1) node (n) {4};
	\draw (cons3) -- (n);
	\path (cons3) +(1,0) node (cons4) {T};
	\draw (cons3) -- (cons4);
	}
	\onslide<6->{
	\begin{scope}[color=gray]
	\path (cons2) +(1,0) node (cons3) {:};
	\draw (cons2) -- (cons3);
	\path (cons3) +(0,-1) node (n) {4};
	\draw (cons3) -- (n);
	\path (cons3) +(1,0) node (cons4) {$\cdots$};
	\draw (cons3) -- (cons4);
	\end{scope}
	}
	}
\draw (comma) -- (t1);
\draw (comma) -- (t2);
\onslide<1-2>{\draw (t1) -- (f);}
\onslide<3-4>{\draw (t1) -- (lt);}
\onslide<5>{\draw (t1) -- (cons3);}
\draw (t2) -- (f);
\end{tikzpicture}

\vfill
\onslide<7->{works, but fragile -- might be thwarted by compiler optimizations}

\end{frame}

\begin{frame}[fragile,t]
\frametitle{Allow the programmer to copy a thunk: dup}

\begin{haskell}
let xs = [1..100000000]
in (case dup xs of Box xs' -> last xs',
    case dup xs of Box xs' -> length xs')
\end{haskell}
\vfill

\centering
\begin{tikzpicture}
[level/.style={sibling distance=20mm/2^#1,level distance=6mm},
every node/.style={circle,inner sep=1.5pt},
eval/.style={circle,fill=black,color=black},
%every path/.style={->},
]
\path
  (0,0) node (comma) {(,)}
 +(-1,-1) coordinate (ct1)
 +(1,-1) coordinate (ct2)
 +(0,-2) coordinate (cot);
\onslide<1>{\node at (ct1) (t1) {T};}
\onslide<2-5>{\node[eval] at (ct1) (t1) {:};}
\onslide<6->{\node at (ct1) (t1) {1e8};}

\onslide<1->{\node at (ct2) (t2) {T};}

\onslide<1->{\node at (cot) (ot) {T};}
\path (t1) +(1,-2) coordinate (clt);
\onslide<3>{\path (clt) node (lt) {T};
	\draw[double] (lt) -- (ot);}
\onslide<4>{\node[alias=cons0] at (clt) (lt) {:};}
\onslide<5-6>{\node[alias=cons0,color=gray] at (clt) (lt) {:};}
\onslide<4>{
	\path (cons0) +(0,-1) node (n) {1};
	\draw (cons0) -- (n);
	\path (cons0) +(1,0) node (cons1) {T};
	\draw (cons0) -- (cons1);
}
\onslide<5-6>{
	\begin{scope}[color=gray]
	\path (cons0) +(0,-1) node (n) {1};
	\draw (cons0) -- (n);
	\path (cons0) +(1,0) node (cons1) {:};
	\draw (cons0) -- (cons1);

	\path (cons1) +(0,-1) node (n) {2};
	\draw (cons1) -- (n);
	\path (cons1) +(1,0) node (cons2) {:};
	\draw (cons1) -- (cons2);

	\path (cons2) +(0,-1) node (n) {3};
	\draw (cons2) -- (n);
	\end{scope}

	\onslide<5>{
	\path (cons2) +(1,0) node (cons3) {:};
	\draw[color=gray] (cons2) -- (cons3);
	\path (cons3) +(0,-1) node (n) {4};
	\draw (cons3) -- (n);
	\path (cons3) +(1,0) node (cons4) {T};
	\draw (cons3) -- (cons4);
	}
	\onslide<6>{
	\begin{scope}[color=gray]
	\path (cons2) +(1,0) node (cons3) {:};
	\draw (cons2) -- (cons3);
	\path (cons3) +(0,-1) node (n) {4};
	\draw (cons3) -- (n);
	\path (cons3) +(1,0) node (cons4) {$\cdots$};
	\draw (cons3) -- (cons4);
	\end{scope}
	}
	}
\draw (comma) -- (t1);
\draw (comma) -- (t2);
\onslide<1-2>{\draw (t1) -- (ot);}
\onslide<3-4>{\draw (t1) -- (lt);}
\onslide<5>{\draw (t1) -- (cons3);}
\draw (t2) -- (ot);
\end{tikzpicture}

\hide
\onslide<7->
the consumer, not the generator, controls sharing. no code restructuring.

\end{frame}

\begin{frame}
\frametitle{The sledgehammer: deepDup}

\begin{columns}[T,onlytextwidth]
\begin{column}{.3\linewidth}
morally, \lstinline-deepDup x- copies the whole heap reachable by x
\end{column}
\begin{column}{.3\linewidth}
\centering
\begin{tikzpicture}
[
every node/.style={circle,inner sep=1.3pt},
eval/.style={circle,fill=black,color=black},
%every path/.style={->},
]
\path (0,0) node[eval] (eval) {:};
\path<1> (0,-1) node (d1) {D};
\path (0,-2) node (t1) {T};
\path<2> (1,-2) node (t1') {T};
\path<3-> (1,-2) node (t1') {C};
\path<2-3> (1,-3) node (d2) {D};
\path (0,-4) node (comma) {(,)};
\path (-1,-5) node (t2) {T};
\path (2,-5) node (t3) {T};
\path<4-> (1,-4) node (comma') {(,)};
\path<4> (0,-4.5) node (d3) {D};
\path<4-6> (1.5,-4.5) node (d4) {D};
\path<5> (0,-5) node (t2') {C};
\path<6-> (0,-5) node (t2') {C};
\path<7> (1,-5) node (t3') {T};
\path<8-> (1,-5) node (t3') {I};

\draw<1> (eval) -- (d1);
\draw<1> (d1) -- (t1);
\draw (t1) -- (comma);
\draw<2-> (eval) -- (t1');
\draw<2-3> (t1') -- (d2);
\draw<2-3> (d2) -- (comma);
\draw<2>[double] (t1) -- (t1');
\draw<4-> (t1') -- (comma');
\draw (comma) -- (t2);
\draw (comma) -- (t3);
\draw<4> (comma') -- (d3);
\draw<4> (d3) -- (t2);
\draw<5-> (comma') -- (t2');
\draw<5>[double] (t2) -- (t2');
\draw<4-6> (comma') -- (d4);
\draw<4-6> (d4) -- (t3);
\draw<7-> (comma') -- (t3');
\draw<7>[double] (t3) -- (t3');

\begin{scope}[on background layer,every path/.style={
line width=.7cm, line cap=round, KITgreen50, line join = round
}]
\draw<1> (eval.center) -- (d1.center);
\draw<2-3> (eval.center) -- (t1'.center) -- (d2.center);
\draw<4> (eval.center) -- (t1'.center) -- (d2.center) -- (comma'.center) -- (d3.center)
	(comma'.center) -- (d4.center);
\draw<5-6> (eval.center) -- (t1'.center) -- (d2.center) -- (comma'.center) -- (t2'.center)
	(comma'.center) -- (d4.center);
\draw<7-> (eval.center) -- (t1'.center) -- (d2.center) -- (comma'.center) -- (t2'.center)
	(comma'.center) -- (t3'.center);
\end{scope}

\path[use as bounding box] (-1.5,-5.5) (2.5,0);
\end{tikzpicture}
\end{column}
\begin{column}{.3\linewidth}
\onslide<9>{%
really, \lstinline-deepDup x- copies the whole heap reachable by x lazily
}
\end{column}
\end{columns}

\end{frame}

\newcommand{\mdup}{\text{\textsf{dup}}}
\newcommand{\mdeepDup}{\text{\textsf{deepDup}}}
\newcommand{\sVar}{\text{Var}}
\newcommand{\sExp}{\text{Exp}}
\newcommand{\sHeap}{\text{Heap}}
\newcommand{\sVal}{\text{Val}}
\newcommand{\sValue}{\text{Value}}
\newcommand{\sEnv}{\text{Env}}
\newcommand{\sApp}[2]{\operatorname{#1}#2}
\newcommand{\sLam}[2]{\text{\textlambda} #1.\, #2}
\newcommand{\sDup}[1]{\sApp \mdup #1}
\newcommand{\sDeepDup}[1]{\sApp \mdeepDup #1}
\newcommand{\sLet}[2]{\text{\textsf{let}}\ #1\ \text{\textsf{in}}\ #2}
\newcommand{\sred}[4]{#1 : #2 \Downarrow #3 : #4}
\newcommand{\sRule}[1]{\text{{\textsc{#1}}}}
\newcommand{\fv}[1]{\text{fv}(#1)}
\newcommand{\ufv}[1]{\text{ufv}(#1)}
\newcommand{\ur}[2]{\text{ur}_{#1}(#2)}
\newcommand{\dom}[1]{\text{dom}\,#1}
\newcommand{\fresh}[1]{#1'}


\begin{frame}
\frametitle{Comes with proofs included.}

\begin{mathpar}
\inferrule
{\sred{\Gamma,x\mapsto e, \fresh x\mapsto \hat e} {\fresh x} \Delta z \\ \fresh x \text{ fresh}}
{\sred{\Gamma,x\mapsto e}{\sDup x} \Delta z}
\sRule{Dup}
\end{mathpar}
\vfill
\begin{mathpar}
\inferrule
{
\sred{
\Gamma,
x\mapsto e,
\begin{array}[b]{l}
\fresh x\mapsto \hat e[\fresh y_1/y_1,\ldots, \fresh y_n/y_n],
\\
\fresh y_1 \mapsto \sDeepDup y_1,\ldots,
\fresh y_n \mapsto \sDeepDup y_n
\end{array}
} {\fresh x} \Delta z 
\\
\ufv e = \{y_1,\ldots,y_n\}
\\
\fresh x,\ \fresh y_1,\ldots,y_n \text{ fresh}
}
{\sred{\Gamma,x\mapsto e}{\sDeepDup x} \Delta z}
\sRule{Deep}
\end{mathpar}
\begin{center}
\vfill
\vfill
(based on Launchbury’s „A natural seantics for lazy evaluation“)
\end{center}
\end{frame}
\begin{frame}
\frametitle{Where to read more}
See\par {\centering\Large \url{http://arxiv.org/abs/1207.2017}\par} for
\begin{itemize}
\item more motiviation,
\item benchmarked comparison with other approaches to avoid sharing,
\item semantics and proofs,
\item details on the implementation and
\item a description of current shortcomings.
\end{itemize}
\vfill
See\par {\centering\Large \url{http://darcs.nomeata.de/ghc-dup}\par} for
\begin{itemize}
\item the code.
\end{itemize}

\end{frame}


\begin{frame}[fragile]
\frametitle{A related, younger idea}

\begin{haskell}
import GHC.Prim (noupdate)

let xs = noupdate [1..100000000]
in (last xs, length xs)
\end{haskell}
\vfill

\centering
For a thunk wrapped in\\[0.5em]
{\Large \lstinline!noupdate :: a -> a!},\\[0.5em]
no blackhole and no update frame is created\\[0.5em]
$\Longrightarrow$ sharing is effectively prevented.
\vfill

(Ask me for my ghc branch.)

\end{frame}


\begin{frame}[fragile]
\frametitle{Also nice: ghc-vis}

\centering
\textit{Demonstration}
\vfill
see\\
\url{http://hackage.haskell.org/package/ghc-vis}\\
and\\
\url{http://felsin9.de/nnis/ghc-vis/}
\vfill

\end{frame}



\end{document}
